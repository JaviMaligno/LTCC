\documentclass{article}
\usepackage{amsmath,accents}%
\usepackage{amsfonts}%
\usepackage{amssymb}%
\usepackage{comment}
\usepackage{graphicx}
\usepackage{mathrsfs}
\usepackage[utf8]{inputenc}
\usepackage{alphabeta}
%\usepackage[T1,LGR]{fontenc}
%\usepackage{hyperref}
\usepackage[colorlinks = true,
            linkcolor = blue,
            urlcolor  = blue,
            citecolor = blue,
            anchorcolor = blue]{hyperref}
\usepackage{amsfonts}
\usepackage{amssymb}
\usepackage{graphicx}
\usepackage{mathrsfs}
\usepackage{setspace}  
\usepackage{amsthm}
\usepackage{nccmath}
\usepackage[UKenglish]{babel}
\usepackage{multirow}
\usepackage{enumerate}
\usepackage{listings}
\usepackage{tkz-graph}  
    \usetikzlibrary{shapes.geometric}%  
    
\theoremstyle{plain}

\renewcommand{\baselinestretch}{1,4}
\setlength{\oddsidemargin}{0.5in}
\setlength{\evensidemargin}{0.5in}
\setlength{\textwidth}{5.4in}
\setlength{\topmargin}{-0.25in}
\setlength{\headheight}{0.5in}
\setlength{\headsep}{0.6in}
\setlength{\textheight}{8in}
\setlength{\footskip}{0.75in}

\newtheorem{theorem}{Teorema}[section]
\newtheorem{acknowledgement}{Acknowledgement}
\newtheorem{algorithm}{Algorithm}
\newtheorem{axiom}{Axiom}
\newtheorem{case}{Case}
\newtheorem{claim}{Claim}
\newtheorem{propi}[theorem]{Propiedades}
\newtheorem{condition}{Condition}
\newtheorem{conjecture}{Conjecture}
\newtheorem{coro}[theorem]{Corolario}
\newtheorem{criterion}{Criterion}
\newtheorem{defi}[theorem]{Definición}
\newtheorem{example}[theorem]{Ejemplo}

\theoremstyle{definition}
\newtheorem{exercise}{Exercise}
\newtheorem{lemma}[theorem]{Lema}
\newtheorem{nota}[theorem]{Nota}
\newtheorem{sol}{Solución}
\newtheorem*{sol*}{Solution}
\newtheorem{prop}[theorem]{Proposición}
\newtheorem{remark}{Remark}

\newtheorem{dem}[theorem]{Demostración}

\newtheorem{summary}{Summary}

\providecommand{\abs}[1]{\lvert#1\rvert}
\providecommand{\norm}[1]{\lVert#1\rVert}
\providecommand{\ninf}[1]{\norm{#1}_\infty}
\providecommand{\numn}[1]{\norm{#1}_1}
\providecommand{\gabs}[1]{\left|{#1}\right|}
\newcommand{\bor}[1]{\mathcal{B}(#1)}
\newcommand{\R}{\mathbb{R}}
\newcommand{\Q}{\mathbb{Q}}
\newcommand{\Z}{\mathbb{Z}}
\newcommand{\F}{\mathbb{F}}
\newcommand{\X}{\chi}
\providecommand{\Zn}[1]{\Z / \Z #1}
\newcommand{\resi}{\varepsilon_L}
\newcommand{\cee}{\mathbb{C}}
\providecommand{\conv}[1]{\overset{#1}{\longrightarrow}}
\providecommand{\gene}[1]{\langle{#1}\rangle}
\providecommand{\convcs}{\xrightarrow{CS}}
% xrightarrow{d}[d]
\setcounter{exercise}{0}
\newcommand{\cicl}{\mathcal{C}}

\newenvironment{ejercicio}[2][Estado]{\begin{trivlist}
\item[\hskip \labelsep {\bfseries Ejercicio}\hskip \labelsep {\bfseries #2.}]}{\end{trivlist}}
%--------------------------------------------------------
\begin{document}

\title{Applied Bayesian Methods - Week 3 Exercises}
\author{Javier Aguilar Martín}
\date{\today}
\maketitle
\begin{exercise}
Construct a DAG to represent the following joint probability distribution:
\[p(A,B,C,D,E, F) = p(A)p(B)p(C|A,B)p(D|B)p(E|B,C,D)p(F|C,E).\]
\end{exercise}
\begin{sol*}\

 \tikzstyle{VertexStyle} = [shape            = circle,
                               minimum width    = 6ex,%
                               draw]

    \tikzstyle{EdgeStyle}   = [->,>=stealth']      

    \begin{tikzpicture}[scale=1.5] 
    \SetGraphUnit{3} 
    \Vertex{A}  
  	\SO(A){B}\EA(A){C}\EA(B){D}\EA(C){E}\EA(D){F}
  	\Edges(A,C, E, F) \Edges(C,F) \Edges(B,D,E) \Edges(B,C)\Edges(B,E)
    \end{tikzpicture}
\end{sol*}

\begin{exercise}
Considering the following hierarchical model:

\begin{align*}
Y_i \mid \theta_i &\sim \mathrm{NegBin}(r,\theta_i),& i = 1, 2\\
\theta_i\mid \alpha,\beta & \sim \mathrm{Beta}(\alpha,\beta), & i=1,2\\
\alpha &\sim \mathrm{Exponential}(0.01)& \\
\beta & \sim \mathrm{Exponential}(0.01) &
\end{align*}
where $r$ is a constant; $\alpha$; $\alpha$, $\beta$, $\theta_1$ and $\theta_2$ are unobserved random variables; the observed values of $Y_1$ and $Y_2$ are $y_1$ and $y_2$, respectively.
\begin{enumerate}[(a)]
\item Draw the DAG for this model.
\item Show that the full-conditional distribution of $\theta_1$ is a Beta distribution.
\item Suppose that you have worked out the full-conditional distributions
of all four unobserved random variables $\alpha$, $\beta$, $\theta
_1$ and $\theta_2$. Explain in detail how to use these distributions in the Gibbs sampler to obtain a sample from the posterior distribution $p(\theta_2 \mid y)$.
\end{enumerate}
The Negative Binomial distribution, $\mathrm{NegBin}(r,\theta)$, has probability mass function
\[p(Y = y \mid \theta) = \binom{y + r - 1}{y}\theta^r(1 - \theta)^y,\ y = 0, 1, 2, \dots\]

\end{exercise}
\begin{sol*}\
\begin{enumerate}[(a)]
\item The DAG looks as follows.

\begin{tikzpicture}[scale=2,line cap=round,line join=round,>=triangle 45,x=1.0cm,y=1.0cm]
\clip(0.02333333333333283,0.1) rectangle (3.854444444444445,3.0333333333333274);
\draw(1.6188888888888886,0.8777777777777761) -- (1.014444444444444,0.8822222222222205) -- (1.01,0.2777777777777763) -- (1.6144444444444441,0.2733333333333318) -- cycle;
\draw(2.6,2.) -- (0.8,2.) -- (0.8,0.2) -- (2.6,0.2) -- cycle;
\draw(2.3877777777777776,0.8733333333333316) -- (1.81,0.8688888888888888) -- (1.8144444444444425,0.2911111111111115) -- (2.39222222222222,0.29555555555555396) -- cycle;
\draw(1.3744444444444446,2.5933333333333293) circle (0.3340529269811789cm);
\draw(2.174444444444444,2.5933333333333293) circle (0.3489048120924053cm);
\draw(1.2277777777777774,1.602222222222219) circle (0.26666666666666655cm);
\draw (1.6188888888888886,0.8777777777777761)-- (1.014444444444444,0.8822222222222205);
\draw (1.014444444444444,0.8822222222222205)-- (1.01,0.2777777777777763);
\draw (1.01,0.2777777777777763)-- (1.6144444444444441,0.2733333333333318);
\draw (1.6144444444444441,0.2733333333333318)-- (1.6188888888888886,0.8777777777777761);
\draw (2.6,2.)-- (0.8,2.);
\draw (0.8,2.)-- (0.8,0.2);
\draw (0.8,0.2)-- (2.6,0.2);
\draw (2.6,0.2)-- (2.6,2.);
\draw [->] (1.348212114330385,2.26031197834017) -- (1.1655555555555552,1.8555555555555518);
\draw [->] (2.0448944571415244,2.269371346842553) -- (1.29,1.8644444444444408);
\draw [->] (1.2470509139320365,1.336252943293448) -- (1.2749419568822549,0.8803067993366483);
\draw (1.25,2.7) node[anchor=north west] {$\alpha$};
\draw (2.081111111111111,2.753333333333328) node[anchor=north west] {$\beta$};
\draw (1.094444444444444,1.7666666666666633) node[anchor=north west] {$\theta_1$};
\draw(2.2,1.6) circle (0.25706078447242453cm);
\draw (2.3877777777777776,0.8733333333333316)-- (1.81,0.8688888888888888);
\draw (1.81,0.8688888888888888)-- (1.8144444444444425,0.2911111111111115);
\draw (1.8144444444444425,0.2911111111111115)-- (2.39222222222222,0.29555555555555396);
\draw (2.39222222222222,0.29555555555555396)-- (2.3877777777777776,0.8733333333333316);
\draw [->] (2.183270705901152,1.3434841571510008) -- (2.1166666666666667,0.8866666666666649);
\draw [->] (1.5309491192558253,2.2982102322604434) -- (2.018230576122415,1.781769423877585);
\draw [->] (2.206032801905055,2.245861401266612) -- (2.227777777777778,1.8555555555555518);
\draw (2.09,1.775555555555552) node[anchor=north west] {$\theta_2$};
\draw (1.1566666666666663,0.7222222222222208) node[anchor=north west] {$Y_1$};
\draw (2.0233333333333334,0.7044444444444431) node[anchor=north west] {$Y_2$};
\end{tikzpicture}
\item The full conditional distribution of $\theta_1$ is
\begin{align*}
p(\theta_1\mid Y_1,Y_2,\theta_2,\alpha,\beta)&=p(\theta_1\mid \alpha,\beta)p(Y_1\mid \theta_1)\\
&\propto \theta_1^{\alpha-1}(1-\theta_1)^{\beta-1}\theta^r(1-\theta)^{y_1}\\
&=\theta_1^{\alpha+r-1}(1-\theta_1)^{\beta+y_1-1}.
\end{align*}
So the full conditional distribution of $\theta_1$ is a $\mathrm{Beta}(\alpha+r-1,\beta+y_1-1)$.
\item This is not in the lecture notes. An explanation in general can be found in \href{https://en.wikipedia.org/wiki/Gibbs_sampling#Implementation}{Wikipedia}.
\end{enumerate}
\end{sol*}




\end{document}