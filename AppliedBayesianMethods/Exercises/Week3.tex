\documentclass{article}
\usepackage{amsmath,accents}%
\usepackage{amsfonts}%
\usepackage{amssymb}%
\usepackage{comment}
\usepackage{graphicx}
\usepackage{mathrsfs}
\usepackage[utf8]{inputenc}
\usepackage{alphabeta}
%\usepackage[T1,LGR]{fontenc}
\usepackage{amsfonts}
\usepackage{amssymb}
\usepackage{graphicx}
\usepackage{mathrsfs}
\usepackage{setspace}  
\usepackage{amsthm}
\usepackage{nccmath}
\usepackage[UKenglish]{babel}
\usepackage{multirow}
\usepackage{enumerate}
\usepackage{listings}

\theoremstyle{plain}

\renewcommand{\baselinestretch}{1,4}
\setlength{\oddsidemargin}{0.5in}
\setlength{\evensidemargin}{0.5in}
\setlength{\textwidth}{5.4in}
\setlength{\topmargin}{-0.25in}
\setlength{\headheight}{0.5in}
\setlength{\headsep}{0.6in}
\setlength{\textheight}{8in}
\setlength{\footskip}{0.75in}

\newtheorem{theorem}{Teorema}[section]
\newtheorem{acknowledgement}{Acknowledgement}
\newtheorem{algorithm}{Algorithm}
\newtheorem{axiom}{Axiom}
\newtheorem{case}{Case}
\newtheorem{claim}{Claim}
\newtheorem{propi}[theorem]{Propiedades}
\newtheorem{condition}{Condition}
\newtheorem{conjecture}{Conjecture}
\newtheorem{coro}[theorem]{Corolario}
\newtheorem{criterion}{Criterion}
\newtheorem{defi}[theorem]{Definición}
\newtheorem{example}[theorem]{Ejemplo}

\theoremstyle{definition}
\newtheorem{exercise}{Exercise}
\newtheorem{lemma}[theorem]{Lema}
\newtheorem{nota}[theorem]{Nota}
\newtheorem{sol}{Solución}
\newtheorem*{sol*}{Solution}
\newtheorem{prop}[theorem]{Proposición}
\newtheorem{remark}{Remark}

\newtheorem{dem}[theorem]{Demostración}

\newtheorem{summary}{Summary}

\providecommand{\abs}[1]{\lvert#1\rvert}
\providecommand{\norm}[1]{\lVert#1\rVert}
\providecommand{\ninf}[1]{\norm{#1}_\infty}
\providecommand{\numn}[1]{\norm{#1}_1}
\providecommand{\gabs}[1]{\left|{#1}\right|}
\newcommand{\bor}[1]{\mathcal{B}(#1)}
\newcommand{\R}{\mathbb{R}}
\newcommand{\Q}{\mathbb{Q}}
\newcommand{\Z}{\mathbb{Z}}
\newcommand{\F}{\mathbb{F}}
\newcommand{\X}{\chi}
\providecommand{\Zn}[1]{\Z / \Z #1}
\newcommand{\resi}{\varepsilon_L}
\newcommand{\cee}{\mathbb{C}}
\providecommand{\conv}[1]{\overset{#1}{\longrightarrow}}
\providecommand{\gene}[1]{\langle{#1}\rangle}
\providecommand{\convcs}{\xrightarrow{CS}}
% xrightarrow{d}[d]
\setcounter{exercise}{0}
\newcommand{\cicl}{\mathcal{C}}

\newenvironment{ejercicio}[2][Estado]{\begin{trivlist}
\item[\hskip \labelsep {\bfseries Ejercicio}\hskip \labelsep {\bfseries #2.}]}{\end{trivlist}}
%--------------------------------------------------------
\begin{document}

\title{Applied Bayesian Methods - Week 3 Exercises}
\author{Javier Aguilar Martín}
\date{\today}
\maketitle
\begin{exercise}
Suppose that
$Y | θ \sim \mathrm{Normal}(θ, τ-1)$,
where $θ$ is an unknown parameter and the value of $τ$ is known.
\begin{enumerate}[(a)]
\item Explain briefly what a ``natural conjugate prior'' means.
\item List any two advantages of using conjugate priors.
\item Show that the distribution of $Y | θ$ belongs to the one-parameter
exponential family.
\item Using your knowledge of the exponential family or otherwise, derive
the conjugate prior for $θ$ and identify the mean and precision
of $θ$ for this prior distribution.
\item Is this prior a natural conjugate prior? Justify your answer.

\end{enumerate}
Suppose we thought that it was reasonable to use a Normal prior distribution
for $θ$:

\[θ \sim \mathrm{Normal}(μ_0, φ^{-1}_0 ) ;\]
and we believed that there was a probability of 0.95 that the values of
$θ$ lay symmetrically between 1 and 3.
\begin{enumerate}
\item[(f)] Using this information to specify values for $μ_0$ and $φ_0$.
\item[(g)] Since then, we have collected a random sample of $n$ independent
observations, $Y_1,\dots, Y_n$, of $Y$. Let $y = (y_1, \dots , y_n)$ denote the values of these observations, and $\bar{y} = \frac{1}{n}Σ^n_{i=1} y_i$ denote the sample mean. Derive the posterior distribution of $θ$.
\item[(h)] Given $n = 10$, $\bar{y} = 2.5$ and $τ = 1$, calculate the 95\% posterior central credible interval of $θ$.
\item[(i)] Comment briefly on the change in our belief, before and after we
see the data $y$, about the mean and variance of $θ$.
\end{enumerate}
 

\end{exercise}
\begin{sol*}\
\begin{enumerate}[(a)]
\item A \emph{natural conjugate prior} is a conjugate prior that also belongs to the same class as the likelihood as a function of $\theta$.
\item They are convenient for calculations and ensure that the posterior follows a known parametric form. 
\item Since $Y\mid\theta\sim\mathrm{Normal}(\theta,\tau-1)$, it has a density  $\phi$ given by
\[
\phi(y)=\frac{1}{(\tau-1)\sqrt{2\pi}}e^{-\frac{1}{2}(\frac{y-\theta}{\tau-1})^2}.
\]
The trick here is considering functions from higher dimensional spaces and using dot product. More precisely, $h(\theta)=(\frac{\theta}{\tau-1},-\frac{1}{2(\tau-1)^2})$, $t(y)=(y,y^2)$, $f(y)=\frac{1}{\sqrt{2\pi}}$ and $g(\theta)= e^{-\frac{\theta^2}{2(\tau-1)^2}}+\frac{1}{\tau-1}$.
\end{enumerate}
\end{sol*}

\begin{exercise}
Suppose $Y \sim \mathrm{Normal}(θ, τ-1)$ where $θ$ is known and $τ$ is unknown.
\begin{enumerate}[(a)]
\item Show that this likelihood belongs to a one-parameter exponential
family.
\item Find the conjugate prior family for $τ$.
\end{enumerate}
\end{exercise}



\begin{exercise}

The Negative Binomial distribution, $\mathrm{NegBin}(r, θ)$, describes the distribution of the number of failures before the $r$-th success in an experiment that consists of a sequence of independent and identically distributed (i.i.d.) Bernoulli trials, where each trial has a probability $θ$ of success. It has probability mass function
\[p(Y = y | θ) =\binom{y + r - 1}{y}θ^r(1 - θ)^y ,\ y = 0, 1, 2, \dots ,\]
with mean $\frac{r(1-θ)}{θ}$. Now suppose that $Y \sim \mathrm{NegBin}(r, θ)$, where $θ$ is unknown and the value of $r$ is known.
\begin{enumerate}[(a)]
\item Derive Jeffreys’ prior for $θ$. Show that it can be written as a Beta
distribution.
\item Is this Jeffreys’ prior proper? Justify briefly your answer.
\item Suppose that we carry out independently the experiment for $n$
times, and each time the value of $Y$ is recorded as $y_i$, $i = 1, \dots , n$. Let $y$ denote the values $(y_1, \dots , y_n)$ of the sample and $\bar{y}$ denote the sample mean $\bar{y} = \frac{1}{n}Σ^n_{i=1} y_i$. Using Jeffreys’ prior from (3a) and this sample, show that the posterior distribution of $θ$ is a Beta distribution. 
\end{enumerate}
\end{exercise}
\begin{sol*}\

\end{sol*}
\end{document}