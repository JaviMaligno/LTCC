\documentclass{article}
\usepackage{amsmath,accents}%
\usepackage{amsfonts}%
\usepackage{amssymb}%
\usepackage{comment}
\usepackage{graphicx}
\usepackage{mathrsfs}
\usepackage[utf8]{inputenc}
\usepackage{amsfonts}
\usepackage{amssymb}
\usepackage{graphicx}
\usepackage{mathrsfs}
\usepackage{setspace}  
\usepackage{amsthm}
\usepackage{nccmath}
\usepackage[UKenglish]{babel}
\usepackage{multirow}
\usepackage{enumerate}
\usepackage{listings}
\usepackage{hyperref}

\theoremstyle{plain}

\renewcommand{\baselinestretch}{1,4}
\setlength{\oddsidemargin}{0.5in}
\setlength{\evensidemargin}{0.5in}
\setlength{\textwidth}{5.4in}
\setlength{\topmargin}{-0.25in}
\setlength{\headheight}{0.5in}
\setlength{\headsep}{0.6in}
\setlength{\textheight}{8in}
\setlength{\footskip}{0.75in}

\newtheorem{theorem}{Teorema}[section]
\newtheorem{acknowledgement}{Acknowledgement}
\newtheorem{algorithm}{Algorithm}
\newtheorem{axiom}{Axiom}
\newtheorem{case}{Case}
\newtheorem{claim}{Claim}
\newtheorem{propi}[theorem]{Propiedades}
\newtheorem{condition}{Condition}
\newtheorem{conjecture}{Conjecture}
\newtheorem{coro}[theorem]{Corolario}
\newtheorem{criterion}{Criterion}
\newtheorem{defi}[theorem]{Definición}
\newtheorem{example}[theorem]{Ejemplo}

\theoremstyle{definition}
\newtheorem{exercise}{Exercise}
\newtheorem{lemma}[theorem]{Lema}
\newtheorem{nota}[theorem]{Nota}
\newtheorem{sol}{Solución}
\newtheorem*{sol*}{Solution}
\newtheorem{prop}[theorem]{Proposición}
\newtheorem{remark}{Remark}

\newtheorem{dem}[theorem]{Demostración}

\newtheorem{summary}{Summary}

\providecommand{\abs}[1]{\lvert#1\rvert}
\providecommand{\norm}[1]{\lVert#1\rVert}
\providecommand{\ninf}[1]{\norm{#1}_\infty}
\providecommand{\numn}[1]{\norm{#1}_1}
\providecommand{\gabs}[1]{\left|{#1}\right|}
\newcommand{\bor}[1]{\mathcal{B}(#1)}
\newcommand{\R}{\mathbb{R}}
\newcommand{\Q}{\mathbb{Q}}
\newcommand{\Z}{\mathbb{Z}}
\newcommand{\F}{\mathbb{F}}
\newcommand{\X}{\chi}
\providecommand{\Zn}[1]{\Z / \Z #1}
\newcommand{\resi}{\varepsilon_L}
\newcommand{\cee}{\mathbb{C}}
\providecommand{\conv}[1]{\overset{#1}{\longrightarrow}}
\providecommand{\gene}[1]{\langle{#1}\rangle}
\providecommand{\convcs}{\xrightarrow{CS}}
% xrightarrow{d}[d]
\setcounter{exercise}{0}
\newcommand{\cicl}{\mathcal{C}}

\newenvironment{ejercicio}[2][Estado]{\begin{trivlist}
\item[\hskip \labelsep {\bfseries Ejercicio}\hskip \labelsep {\bfseries #2.}]}{\end{trivlist}}
%--------------------------------------------------------
\begin{document}

\title{Kernel Methods in Statistics and Machine Learning - Week 1 Exercises }
\author{Javier Aguilar Martín}
\date{\today}
\maketitle
\begin{exercise}
Show that the Gaussian kernel 
\[
k(x,y) = \exp\left(-\frac{|x-y|^2}{\sigma^2}\right)
\]
is positive definite.

\end{exercise}
\begin{sol*}
We start writing $k$ as
\[
k(x,y) = \exp\left(-\frac{|x|^2}{\sigma^2}\right)\exp\left(\frac{2xy}{\sigma^2}\right)\exp\left(-\frac{|y|^2}{\sigma^2}\right).
\]
Since the left and right factors are the same function of one variable, this reduces the problem to showing that the middle factor is positive definite. To do that, we use the taylor expansion of the exponential function. This expansion is a the limite of sums with positive coefficients of powers of polynomial kernels. By closure under limits, powers and products we obtain the result. (See \url{http://www0.cs.ucl.ac.uk/staff/m.pontil/reading/haussler.pdf} proof of Theorem 4). Note that a convergent sum with positive of positive definite matrices is again positive definite, since by convergence we can distribute products over the sum, and if all terms are positive the sum is positive as well.

\end{sol*}



\end{document}