\documentclass{article}
\usepackage{amsmath,accents}%
\usepackage{amsfonts}%
\usepackage{amssymb}%
\usepackage{comment}
\usepackage{graphicx}
\usepackage{mathrsfs}
\usepackage[utf8]{inputenc}
\usepackage{amsfonts}
\usepackage{amssymb}
\usepackage{graphicx}
\usepackage{mathrsfs}
\usepackage{setspace}  
\usepackage{amsthm}
\usepackage{nccmath}
\usepackage[UKenglish]{babel}
\usepackage{multirow}
\usepackage{enumerate}
\usepackage{listings}
\usepackage{pgf,tikz}
\usetikzlibrary{arrows}
\theoremstyle{plain}

\renewcommand{\baselinestretch}{1,4}
\setlength{\oddsidemargin}{0.5in}
\setlength{\evensidemargin}{0.5in}
\setlength{\textwidth}{5.4in}
\setlength{\topmargin}{-0.25in}
\setlength{\headheight}{0.5in}
\setlength{\headsep}{0.6in}
\setlength{\textheight}{8in}
\setlength{\footskip}{0.75in}

\newtheorem{theorem}{Teorema}[section]
\newtheorem{acknowledgement}{Acknowledgement}
\newtheorem{algorithm}{Algorithm}
\newtheorem{axiom}{Axiom}
\newtheorem{case}{Case}
\newtheorem{claim}{Claim}
\newtheorem{propi}[theorem]{Propiedades}
\newtheorem{condition}{Condition}
\newtheorem{conjecture}{Conjecture}
\newtheorem{coro}[theorem]{Corolario}
\newtheorem{criterion}{Criterion}
\newtheorem{defi}[theorem]{Definición}
\newtheorem{example}[theorem]{Ejemplo}

\theoremstyle{definition}
\newtheorem{exercise}{Exercise}
\newtheorem{lemma}[theorem]{Lema}
\newtheorem{nota}[theorem]{Nota}
\newtheorem{sol}{Solución}
\newtheorem*{sol*}{Solution}
\newtheorem{prop}[theorem]{Proposición}
\newtheorem{remark}{Remark}

\newtheorem{dem}[theorem]{Demostración}

\newtheorem{summary}{Summary}

\providecommand{\abs}[1]{\lvert#1\rvert}
\providecommand{\norm}[1]{\lVert#1\rVert}
\providecommand{\ninf}[1]{\norm{#1}_\infty}
\providecommand{\numn}[1]{\norm{#1}_1}
\providecommand{\gabs}[1]{\left|{#1}\right|}
\newcommand{\bor}[1]{\mathcal{B}(#1)}
\newcommand{\R}{\mathbb{R}}
\newcommand{\Q}{\mathbb{Q}}
\newcommand{\Z}{\mathbb{Z}}
\newcommand{\F}{\mathbb{F}}
\newcommand{\X}{\chi}
\providecommand{\Zn}[1]{\Z / \Z #1}
\newcommand{\resi}{\varepsilon_L}
\newcommand{\cee}{\mathbb{C}}
\providecommand{\conv}[1]{\overset{#1}{\longrightarrow}}
\providecommand{\gene}[1]{\langle{#1}\rangle}
\providecommand{\convcs}{\xrightarrow{CS}}
% xrightarrow{d}[d]
\setcounter{exercise}{0}
\newcommand{\cicl}{\mathcal{C}}

\newenvironment{ejercicio}[2][Estado]{\begin{trivlist}
\item[\hskip \labelsep {\bfseries Ejercicio}\hskip \labelsep {\bfseries #2.}]}{\end{trivlist}}
%--------------------------------------------------------
\begin{document}

\title{Higher Order Networks Exam}
\author{Javier Aguilar Martín}
\date{\today}
\maketitle
\begin{exercise}
\end{exercise}
\begin{sol*}
(d) two.
\end{sol*}

\begin{exercise}
\end{exercise}
\begin{sol*}
(b) that any simplicial complex is closed under the inclusion of the faces
of each simplex.
\end{sol*}

\begin{exercise}
\end{exercise}
\begin{sol*}
(a) a dynamical variable that can be associated to nodes, links, or even
higher dimensional simplices.
\end{sol*}

\begin{exercise}
\end{exercise}
\begin{sol*}
(b) that every link is incident to at most two triangles.
\end{sol*}


\begin{exercise}
\end{exercise}
\begin{sol*}
The following simplicial complex has dimension 2, 10 verticies, 2 connected components and 1 cycle, as required. 

\definecolor{uuuuuu}{rgb}{0.26666666666666666,0.26666666666666666,0.26666666666666666}
\begin{tikzpicture}[line cap=round,line join=round,>=triangle 45,x=1.0cm,y=1.0cm]
\clip(0,0) rectangle (12,5);
\draw[fill=black,fill opacity=0.10000000149011612] (1.,1.) -- (4.,1.) -- (4.,4.) -- (1.,4.) -- cycle;
\draw(8.,1.) -- (10.,1.) -- (11.,2.7320508075688776) -- (10.,4.464101615137755) -- (8.,4.464101615137755) -- (7.,2.732050807568879) -- cycle;
\draw (1.,1.)-- (4.,1.);
\draw (4.,1.)-- (4.,4.);
\draw (4.,4.)-- (1.,4.);
\draw (1.,4.)-- (1.,1.);
\draw (8.,1.)-- (10.,1.);
\draw (10.,1.)-- (11.,2.7320508075688776);
\draw (11.,2.7320508075688776)-- (10.,4.464101615137755);
\draw (10.,4.464101615137755)-- (8.,4.464101615137755);
\draw (8.,4.464101615137755)-- (7.,2.732050807568879);
\draw (7.,2.732050807568879)-- (8.,1.);
\draw (1.,4.)-- (4.,1.);
\begin{scriptsize}
\draw [fill=black] (1.,1.) circle (2.5pt);
\draw[color=black] (1.1,1.2466666666666655) node {$A$};
\draw [fill=black] (4.,1.) circle (2.5pt);
\draw[color=black] (4.1,1.2466666666666655) node {$B$};
\draw [fill=uuuuuu] (4.,4.) circle (2.5pt);
\draw[color=uuuuuu] (4.1,4.246666666666666) node {$C$};
\draw [fill=uuuuuu] (1.,4.) circle (2.5pt);
\draw[color=uuuuuu] (1.1,4.246666666666666) node {$D$};
\draw [fill=black] (8.,1.) circle (2.5pt);
\draw[color=black] (8.1,1.2466666666666655) node {$E$};
\draw [fill=black] (10.,1.) circle (2.5pt);
\draw[color=black] (10.1,1.2466666666666655) node {$F$};
\draw [fill=uuuuuu] (11.,2.7320508075688776) circle (2.5pt);
\draw[color=uuuuuu] (11.1,2.98) node {$G$};
\draw [fill=uuuuuu] (10.,4.464101615137755) circle (2.5pt);
\draw[color=uuuuuu] (10.1,4.713333333333333) node {$H$};
\draw [fill=uuuuuu] (8.,4.464101615137755) circle (2.5pt);
\draw[color=uuuuuu] (8.1,4.713333333333333) node {$I$};
\draw [fill=uuuuuu] (7.,2.732050807568879) circle (2.5pt);
\draw[color=uuuuuu] (7.1,2.98) node {$J$};
\end{scriptsize}
\end{tikzpicture}

The simplices are
\begin{gather*}
\{[A],[B],[C],[D],[E],[F],[G],[H],[I],[J],\\
[A,B], [B,C], [C,D], [D,A],[B,D], [E,F], [F,G], [G,H], [H,I], [I,J], [J,E],\\
[A,B,D], [B,C,D]
 \}
\end{gather*}
\end{sol*}

\begin{exercise}
\end{exercise}
\begin{sol*}\
\begin{enumerate}[(a)]
\item By looking at the adjacenty of the simplices we obtain.
\[
\mathbf{B}_{[1]} = \begin{matrix}
  & [1, 2] &   [1, 3] &    [2, 3] &    [2, 4] &   [3, 4] \\
[1]&  -1     &    -1     &     0     &    0     &    0   \\
[2]&   1     &     0     &    -1     &   -1     &    0   \\
[3]&   0     &     1     &     1     &    0     &   -1   \\
[4]&   0     &     0     &     0     &    1     &    1
\end{matrix},\ \
\mathbf{B}_{[2]} = \begin{matrix}
   & [1, 2, 3] \\
[1,2]&   1     \\
[1,3]&  -1     \\
[2,3]&   1     \\
[2,4]&   0     \\
[3,4]&   0      
\end{matrix}
\]
Recalling that $\mathbf{L}_{[0]} = \mathbf{B}_{[1]}\mathbf{B}_{[1]}^{\top}$, $\mathbf{L}_{[1]}^{down}=\mathbf{B}_{[1]}^{\top}\mathbf{B}_{[1]}$, $\mathbf{L}_{[1]}^{up}=\mathbf{B}_{[2]}\mathbf{B}_{[2]}^{\top}$ and $\mathbf{L}_{[2]}=\mathbf{B}_{[2]}^{\top}\mathbf{B}_{[2]}$ we obtain
\[
\mathbf{L}_{[0]} = \begin{matrix}
  & [1] &   [2] &    [3] &    [4] \\
[1]&  2     &    -1     &     -1     &    0      \\
[2]&   -1     &     3     &    -1     &   -1      \\
[3]&   -     &     -1     &     3     &    -1       \\
[4]&   0     &      -1    &     -1     &    2   
\end{matrix},\]
\[
 \mathbf{L}_{[1]}^{down} = \begin{matrix}
  & [1, 2] &   [1, 3] &    [2, 3] &    [2, 4] &   [3, 4] \\
[1,2]&  2     &    1     &     -1    &    -1     &    0   \\
[1,3]&   1     &     2    &    1     &   0     &    -1   \\
[2,3]&   -1    &     1     &     2     &    1     &   -1   \\
[2, 4]&   -1     &     0     &     1    &    2     &    1\\
[3, 4]&   0     &     -1     &     -1     &    1     &    2
\end{matrix},\ \
\mathbf{L}_{[1]}^{up} = \begin{matrix}
  & [1, 2] &   [1, 3] &    [2, 3] &    [2, 4] &   [3, 4] \\
[1,2]&  1     &    -1     &     1    &    0     &    0   \\
[1,3]&   -1     &    1   &    -1     &   0     &    0   \\
[2,3]&   1    &     -1     &     1     &    0     &   0   \\
[2, 4]&   0    &     0     &     0   &    0     &    0\\
[3, 4]&   0     &     0    &    0     &    0     &    0
\end{matrix}
\]
\[
\mathbf{L}_{[2]}=\begin{matrix}
& [1,2,3]\\
[1,2,3] & 3
\end{matrix}
\]
\item According to the order of the simplices that we have used for the matrices, $f$ can be represented in coordinates as the vector $(1, -1,1,0,0)$. Clearly, this is in the image of $\mathbf{B}_{[2]}^{\top}$, so the cochain represents a gradient flow.
\end{enumerate}
\end{sol*}

\begin{exercise}
\end{exercise}
\begin{sol*}
The spectral gap, or Fiedler eigenvalue $\mu_F$, is the smallest non-zero eigenvalue of a graph Laplacian matrix (applicable to any order and to up and down Laplacians). 

If the density of eigenvalues $\rho(\lambda)$ scales like
\[
\rho(\lambda)\sim \lambda^{d_S/2 -1 }
\]
for $\lambda <<1$ we call $d_S$ the spectral dimension.

These two spectral properties are related in the following way. The Fiedler eigenvalue of a connected network with spectral dimension
can be estimated from
\[
\int_{\lambda<\mu_F}
\rho(\lambda)d\lambda = \frac{1}{N}.
\]
Assuming $\rho(\lambda) \propto \lambda^{d_S/2 - 1}$ we get
\[
\mu_F \propto N^{-2/d_S}
\]
Therefore the spectral gap closes, i.e. $\mu_F\to 0$, as $N \to \infty$. 

However, there are differences between these spectral properties.





On a graph, if there is a spectral gap,
the characteristic temporal scale for the relaxation
to equilibrium is given by 
\[\tau = 1/\mu_F.\]

More generally, in presence of a spectral gap
in the spectrum of $\mathbf{L}^{down}_{[n]}$ and of $\mathbf{L}^{up}_{[n]}$
the irrotational and the solenoidal components
relax to the steady state with characteristic scale
correspondingly are

\[\tau^{[1]} = 1/\mu^{down}_F ,\ \tau^{[2]} = 1/\mu^{up}_F.\] 


On the other had, assuming that the network has a spectral
dimension the relaxation is power-law
rather than exponential
\[
p(t) \propto t^{-d_S/2}.
\]
Additionally, the synchronized phase is not
thermodynamically achieved
for networks with spectral dimension $d_S\leq 4$.


\end{sol*}





\end{document}