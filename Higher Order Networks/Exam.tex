\documentclass{article}
\usepackage{amsmath,accents}%
\usepackage{amsfonts}%
\usepackage{amssymb}%
\usepackage{comment}
\usepackage{graphicx}
\usepackage{mathrsfs}
\usepackage[utf8]{inputenc}
\usepackage{amsfonts}
\usepackage{amssymb}
\usepackage{graphicx}
\usepackage{mathrsfs}
\usepackage{setspace}  
\usepackage{amsthm}
\usepackage{nccmath}
\usepackage[UKenglish]{babel}
\usepackage{multirow}
\usepackage{enumerate}
\usepackage{listings}
\usepackage{pgf,tikz}
\usetikzlibrary{arrows}
\theoremstyle{plain}

\renewcommand{\baselinestretch}{1,4}
\setlength{\oddsidemargin}{0.5in}
\setlength{\evensidemargin}{0.5in}
\setlength{\textwidth}{5.4in}
\setlength{\topmargin}{-0.25in}
\setlength{\headheight}{0.5in}
\setlength{\headsep}{0.6in}
\setlength{\textheight}{8in}
\setlength{\footskip}{0.75in}

\newtheorem{theorem}{Teorema}[section]
\newtheorem{acknowledgement}{Acknowledgement}
\newtheorem{algorithm}{Algorithm}
\newtheorem{axiom}{Axiom}
\newtheorem{case}{Case}
\newtheorem{claim}{Claim}
\newtheorem{propi}[theorem]{Propiedades}
\newtheorem{condition}{Condition}
\newtheorem{conjecture}{Conjecture}
\newtheorem{coro}[theorem]{Corolario}
\newtheorem{criterion}{Criterion}
\newtheorem{defi}[theorem]{Definición}
\newtheorem{example}[theorem]{Ejemplo}

\theoremstyle{definition}
\newtheorem{exercise}{Exercise}
\newtheorem{lemma}[theorem]{Lema}
\newtheorem{nota}[theorem]{Nota}
\newtheorem{sol}{Solución}
\newtheorem*{sol*}{Solution}
\newtheorem{prop}[theorem]{Proposición}
\newtheorem{remark}{Remark}

\newtheorem{dem}[theorem]{Demostración}

\newtheorem{summary}{Summary}

\providecommand{\abs}[1]{\lvert#1\rvert}
\providecommand{\norm}[1]{\lVert#1\rVert}
\providecommand{\ninf}[1]{\norm{#1}_\infty}
\providecommand{\numn}[1]{\norm{#1}_1}
\providecommand{\gabs}[1]{\left|{#1}\right|}
\newcommand{\bor}[1]{\mathcal{B}(#1)}
\newcommand{\R}{\mathbb{R}}
\newcommand{\Q}{\mathbb{Q}}
\newcommand{\Z}{\mathbb{Z}}
\newcommand{\F}{\mathbb{F}}
\newcommand{\X}{\chi}
\providecommand{\Zn}[1]{\Z / \Z #1}
\newcommand{\resi}{\varepsilon_L}
\newcommand{\cee}{\mathbb{C}}
\providecommand{\conv}[1]{\overset{#1}{\longrightarrow}}
\providecommand{\gene}[1]{\langle{#1}\rangle}
\providecommand{\convcs}{\xrightarrow{CS}}
% xrightarrow{d}[d]
\setcounter{exercise}{0}
\newcommand{\cicl}{\mathcal{C}}

\newenvironment{ejercicio}[2][Estado]{\begin{trivlist}
\item[\hskip \labelsep {\bfseries Ejercicio}\hskip \labelsep {\bfseries #2.}]}{\end{trivlist}}
%--------------------------------------------------------
\begin{document}

\title{Higher Order Networks Exam}
\author{Javier Aguilar Martín}
\date{\today}
\maketitle
\begin{exercise}
\end{exercise}
\begin{sol*}
(d) two.
\end{sol*}

\begin{exercise}
\end{exercise}
\begin{sol*}
(b) that any simplicial complex is closed under the inclusion of the faces
of each simplex.
\end{sol*}

\begin{exercise}
\end{exercise}
\begin{sol*}
(a) a dynamical variable that can be associated to nodes, links, or even
higher dimensional simplices.
\end{sol*}

\begin{exercise}
\end{exercise}
\begin{sol*}
(b) that every link is incident to at most two triangles.
\end{sol*}


\begin{exercise}
\end{exercise}
\begin{sol*}
The following simplicial complex has dimension 2, 10 verticies, 2 connected components and 1 cycle, as required. 

\definecolor{uuuuuu}{rgb}{0.26666666666666666,0.26666666666666666,0.26666666666666666}
\begin{tikzpicture}[line cap=round,line join=round,>=triangle 45,x=1.0cm,y=1.0cm]
\clip(0,0) rectangle (12,5);
\draw[fill=black,fill opacity=0.10000000149011612] (1.,1.) -- (4.,1.) -- (4.,4.) -- (1.,4.) -- cycle;
\draw(8.,1.) -- (10.,1.) -- (11.,2.7320508075688776) -- (10.,4.464101615137755) -- (8.,4.464101615137755) -- (7.,2.732050807568879) -- cycle;
\draw (1.,1.)-- (4.,1.);
\draw (4.,1.)-- (4.,4.);
\draw (4.,4.)-- (1.,4.);
\draw (1.,4.)-- (1.,1.);
\draw (8.,1.)-- (10.,1.);
\draw (10.,1.)-- (11.,2.7320508075688776);
\draw (11.,2.7320508075688776)-- (10.,4.464101615137755);
\draw (10.,4.464101615137755)-- (8.,4.464101615137755);
\draw (8.,4.464101615137755)-- (7.,2.732050807568879);
\draw (7.,2.732050807568879)-- (8.,1.);
\draw (1.,4.)-- (4.,1.);
\begin{scriptsize}
\draw [fill=black] (1.,1.) circle (2.5pt);
\draw[color=black] (1.1,1.2466666666666655) node {$A$};
\draw [fill=black] (4.,1.) circle (2.5pt);
\draw[color=black] (4.1,1.2466666666666655) node {$B$};
\draw [fill=uuuuuu] (4.,4.) circle (2.5pt);
\draw[color=uuuuuu] (4.1,4.246666666666666) node {$C$};
\draw [fill=uuuuuu] (1.,4.) circle (2.5pt);
\draw[color=uuuuuu] (1.1,4.246666666666666) node {$D$};
\draw [fill=black] (8.,1.) circle (2.5pt);
\draw[color=black] (8.1,1.2466666666666655) node {$E$};
\draw [fill=black] (10.,1.) circle (2.5pt);
\draw[color=black] (10.1,1.2466666666666655) node {$F$};
\draw [fill=uuuuuu] (11.,2.7320508075688776) circle (2.5pt);
\draw[color=uuuuuu] (11.1,2.98) node {$G$};
\draw [fill=uuuuuu] (10.,4.464101615137755) circle (2.5pt);
\draw[color=uuuuuu] (10.1,4.713333333333333) node {$H$};
\draw [fill=uuuuuu] (8.,4.464101615137755) circle (2.5pt);
\draw[color=uuuuuu] (8.1,4.713333333333333) node {$I$};
\draw [fill=uuuuuu] (7.,2.732050807568879) circle (2.5pt);
\draw[color=uuuuuu] (7.1,2.98) node {$J$};
\end{scriptsize}
\end{tikzpicture}

The simplices are
\begin{gather*}
\{[A],[B],[C],[D],[E],[F],[G],[H],[I],[J],\\
[A,B], [B,C], [C,D], [D,A],[B,D], [E,F], [F,G], [G,H], [H,I], [I,J], [J,E],\\
[A,B,D], [B,C,D]
 \}
\end{gather*}
\end{sol*}

\begin{exercise}
\end{exercise}
\begin{sol*}

\end{sol*}

\begin{exercise}
\end{exercise}
\begin{sol*}

\end{sol*}





\end{document}